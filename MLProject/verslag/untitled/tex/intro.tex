\section{Tackled Questions}

The overal goal of this paper is to see if it is possible to determine  who is walking with their phone machine by using learning techniques. To do so we use data generated by sensors of their phone and based on a set of similar data that has been gathered before.
To achieve this goal answers need to be found to a number of questions.
First it is necessary to understand which features of the acquired data are useful for the detection. Secondly  based on previous studies  it is shown that the data needs to be split up in windows to achieve better results \cite{bao2004activity}. Therefore a good window size and partitioning strategy is needed. Another question that arises is whether and how we need to preprocess the data and filter out the corrupt parts of it. Then a good classifier needs to be chosen to classify the data. Additionally we also need a correct way to assess the accuracy of the chosen classifier.
Finally the implementation for the solution should be easy to use and understand.